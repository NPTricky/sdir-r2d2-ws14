\RequirePackage[l2tabu,orthodox]{nag} % first line of preamble (!)

\documentclass[%
  professionalfonts,%
  xcolor={%
    usenames,%
    dvipsnames,%
    svgnames,%
    table,%
    hyperref%
  }%
]{beamer}

\mode<presentation>
{
  \usetheme{Warsaw}
  \usecolortheme{seahorse}
  \useinnertheme{circles}
  \useoutertheme{smoothbars}
  \usefonttheme{professionalfonts} % clean display of mathematical letters
  \setbeamertemplate{navigation symbols}{} % no navigation symbols
  \addtobeamertemplate{navigation symbols}{}{%
  	\usebeamerfont{footline}%
  	\usebeamercolor[fg]{footline}%
  	{\large\insertframenumber/\inserttotalframenumber}%
  }
}

% encoding & decoding
\usepackage[T1]{fontenc}            
\usepackage[utf8]{inputenc}
\usepackage[english]{babel}

% math
\usepackage{isomath}
\usepackage{amsmath,amssymb,amsthm}
\usepackage{thmtools}
\usepackage{mathtools}

% fonts
%\usepackage{lmodern}
%\usepackage{mathpazo}
%\usepackage{kpfonts}
%\usepackage{mathptmx}
%\usepackage{stix}
%\usepackage{txfonts}
\usepackage{newtxtext,newtxmath}
%\usepackage{libertine} \usepackage[libertine]{newtxmath}

% figures & graphics
\usepackage{graphicx}
\usepackage[%
  usenames,%
  dvipsnames,%
  svgnames,%
  table,%
  hyperref%
]{xcolor}
\usepackage[%
  margin=10pt,%
  format=hang,%
  parskip=5pt,%
  singlelinecheck=false%
]{subfig}
\graphicspath{{content/}{../content/}{../../content/}{../../../content/}{../../../../content/}}
\DeclareGraphicsExtensions{.pdf,.png,.jpg,.gif}

% videos & medien
\usepackage{media9}%
\addmediapath{content/}
\addmediapath{../content/}
\addmediapath{../../content/}
\addmediapath{../../../content/}
\addmediapath{../../../../content/}

\newcommand{\includemovie}[3]{%
{%
\centering%
\includemedia[%
  width=#1,
  height=#2,%
  activate=pagevisible,%
  deactivate=pageclose,%
  addresource=#3,%
  flashvars={%
    src=#3 % same path as in addresource!
    &autoPlay=true % default: false; if =true, automatically starts playback after activation (see option ‘activation)’
    &loop=false % if loop=true, media is played in a loop
    &controlBarAutoHideTimeout=true %  time span before auto-hide
    &muted=true
  }%
]{}{StrobeMediaPlayback.swf}%
}%
}

% tables
\usepackage{booktabs}

% code listings
\usepackage{listings}
\lstdefinestyle{global}{ %
	basicstyle=\ttfamily\scriptsize\color{Black!90}, %
	stringstyle=\itshape\color{Magenta}, %
	showstringspaces=false, %
	keywordstyle={[1]\bfseries\color{MidnightBlue}}, %
	keywordstyle={[2]\bfseries\color{RedOrange}}, %
	keywordstyle={[3]\bfseries\color{Black!90}}, %
	commentstyle=\slshape\color{Blue}, %
	backgroundcolor=\color{Black!2}, %
	numbers=left, % where to put the line-numbers
  numberstyle=\footnotesize, % the size of the fonts that are used for the line-numbers
  stepnumber=1, % the step between two line-numbers. If it is 1 each line will be numbered
  numbersep=6pt, % how far the line-numbers are from the code
  frame=bt,%
  showspaces=false, % show spaces adding particular underscores
  showstringspaces=false, % underline spaces within strings
  showtabs=false, % show tabs within strings adding particular underscores
  rulesepcolor=\color{Black},
  rulecolor=\color{Black!60},
  tabsize=2, % sets default tabsize to 2 spaces
  captionpos=b, % sets the caption-position to bottom
  breaklines=true, % sets automatic line breaking
  breakatwhitespace=false, % sets if automatic breaks should only happen at whitespace
}

\lstdefinelanguage{pseudo}
{
  morekeywords=[1]{return, if, else, true, false, foreach, in}, %
  morekeywords=[2]{generate_rt, generate_random_state, find_nearest_neighbor,generate_state}, %
  morekeywords=[3]{}, %
  sensitive=false, %
  morecomment=[l]{//}, %
  morecomment=[s]{/*}{*/}, %
}

% microtype - last line of preamble
\usepackage{microtype}

% titlepage
\title{RRT - Rapidly-Exploring Random Trees}
\author{Tim Jagla, André Keuns, Anne Reich}
\institute[FIN]{Otto-von-Guericke-Universität Magdeburg}
\date{\today} 
\logo{\includegraphics[width=0.18\textwidth]{4/cse_logo.jpeg}}

\begin{document}
  \begin{frame}
    \titlepage
  \end{frame}
  
  \section{Collision Free Path Planning}
    \begin{frame}
      \huge{\centering{Collision Free Path Planning}}
    \end{frame}
    
    \subsection*{}
      \begin{frame}{Motivation}
        \begin{itemize}
          \item path planning: find a path from location \emph{A} to \emph{B}
          \item example for path planning:
          \begin{itemize}
            \item mobile robot inside a build
            \item shall go to location XY
          \end{itemize}
          \item example extension for collision free path planning:
          \begin{itemize}
            \item avoiding walls and not falling stairs
          \end{itemize}
        \end{itemize}
        
        \begin{figure}[h]
          \includegraphics[width=0.4\textwidth]{4/example_path_planing_0.png}
          \caption{example for path planing [2]}
          \label{fig:example path planing 0}
        \end{figure}
      \end{frame}
    
      \begin{frame}{Simple Example}
        \begin{itemize}
          \item simple general forward search
          \begin{itemize}
            \item state: unvisited, dead, alive
            \item queue, \emph{Q}, with the set of alive states
            \item start loop over \emph{Q}
            \item in each while iteration check next state
            \begin{itemize}
              \item it is the goal, is terminate
              \item otherwise, it tries applying every possible action
             \end{itemize} 
          \end{itemize}
        \end{itemize}
        \begin{figure}[h]
          \includegraphics[width=0.4\textwidth]{4/example_path_planing_1.png}
          \caption{example for path planing [2]}
          \label{fig:example path planing 1}
        \end{figure}
      \end{frame}
      
      \begin{frame}{Algorithms}
        \begin{itemize}
          \item other known collsion free path planning algorithms
          \begin{itemize}
            \item breadth first
            \item deep first
            \item Dijikstra‘s algorithm
            \item A*
            \item backward search
            \item ...
          \end{itemize}
        \end{itemize}
      \end{frame}

      \begin{frame}{Principles}
        \begin{columns}
          \column[c]{.7\textwidth}
          \begin{itemize}
            \item basic ingredients of planning
            \begin{itemize}
              \item state
              \item input
              \item initial and goal states
              \item a criterion: feasibility and/or optimality
              \item a plan
            \end{itemize}
          \end{itemize}
          
          \column[c]{.3\textwidth}
          \begin{figure}[h]
            \includegraphics[width=0.9\textwidth]{4/beans-recipe-book.png}
            \scriptsize\caption{\scriptsize\url{http://www.chrismadden.co.uk/food/beans-recipe-book.gif} (03.02.2015)}
            \label{fig:/determined-challenge-accepte}
          \end{figure}
        \end{columns}
      \end{frame}
    
  \section{Rapidly-Exploring Random Trees}
    \begin{frame}
      \huge{\centering{Rapidly-Exploring Random Trees}}
    \end{frame}
    
    \subsection*{}
      \begin{frame}{Principles}
        \begin{itemize}
          %\begin{footnotesize}
%            \item grows a tree rooted at the starting configuration by using random samples from the search space
%            \item as each sample is drawn, a connection is attempted between it an the nearest state in the tree
%            \item if the connection is feasible, this results in the addition of the new state to the tree
%            \item the probability of expanding an existing state is proportional to the size of its Voronoi region
%            \begin{itemize}
%              \begin{footnotesize}
%                \item as the largest Voronoi regions belong to the states on the frontier of the search = the tree preferentially expands towards large unsearched areas
%              \end{footnotesize}
%            \end{itemize}
            \item is rooted by the starting configuration
            \item grows by using random samples from the search space
            \item it is attempted a connection from the sample to the nearest state in the tree
            \begin{itemize}
              \item is it feasible, append the state to the tree
            \end{itemize}
          %\end{footnotesize}
        \end{itemize}
        \vspace{-5pt}
        \begin{figure}[h]
          \includegraphics[width=0.38\textwidth]{4/rrt_iterations_45.png}
          \includegraphics[width=0.38\textwidth]{4/rrt_iterations_2345.png}
          \vspace{-10pt}
          \caption{\footnotesize rrt with 45 and 2345 iterations [2]}
          \label{fig:rrt_iterations_45_and_2345}
        \end{figure}

      \end{frame}
    
      \begin{frame}{Nice Properties}
        \begin{itemize}
          \item the expansion is heavily biased toward unexplored portions of the state space
          \item the algorithm is relatively simple
          \item the distribution of vertices approaches the sampling distribution, leading to consistent behaviour
          \item is under very general conditions probabilistically complete 
          \item it can incorporated into a wide variety of planning systems
        \end{itemize}
      \end{frame}
      
%      \begin{frame}{Nice Properties}
%        \begin{itemize}
%          \item it can incorporated into a wide variety of planning systems
%          \item it always remains connected, even though the number of edges is minimal
%          \item can be considered as a plat planning module, which can be adapted and incorporated into a wide variety of planning systems
%          \item entire path planning algorithms can be constructed without requiring the ability to steer the system between two prescribed states, which greatly broadens the applicability of RRTs
%        \end{itemize}
%      \end{frame}
      
      \begin{frame}{Challenges of our work}
        \begin{itemize}
          \item implementation from 2D to 6D over 3D
%          \begin{itemize}
%            \item we are went straight to the configuration space of the robot and so cloud some difficulties go away
%          \end{itemize}
          \item determination of the nearest neighbor of an state/point in space to a state in our tree
%          \begin{itemize}
%            \item one way of determination is over the cartesian space, we build us the from our configurations the x-y-z-coordinates in space and calculate the euclidean distance
%            \item the same way we are going by the splitting of edges
%          \end{itemize}
          \item checking of collision by the trajectory of the robot
%          \begin{itemize}
%            \item one way is, we use our available function for the velocity profile and trajectory generation, and test if the path to our goal state is free 
%          \end{itemize}
        \end{itemize}
        
        \begin{figure}[h]
          \includegraphics[width=0.28\textwidth]{4/determined-challenge-accepted-l.png}
          \scriptsize\caption{\scriptsize\url{http://cdn.alltheragefaces.com/img/faces/large/determined-challenge-accepted-l.png} (03.02.2015)}
          \label{fig:/determined-challenge-accepte}
        \end{figure}
        
      \end{frame}
      
      \begin{frame}{Notations}
        \begin{align*}
        T        & \text{ = RRT (tree of vertices)}\\
        C        & \text{ = configuration space of a rigid body or}\\
                 & \text{ ~~ systems of bodies in a world}\\
        T(C)     & \text{ = tanget bundle of the configuration space}\\
        C_{goal} & \text{ = goal region},  C_{goal} \subset C\\
        C_{obs}  & \text{ = obstacle region},  C_{obs} \subset C\\
        C_{free} & \text{ = region without obstacles}, C_{free} \subset C\\
        q_{init} & \text{ = initale state}\\
        q_n      & \text{ = neighbor of a state}\\
        alpha    & \text{ = random state}\\
        edges    & \text{ = correspond to a path that lies entirely in } C_{free}\\
        \end{align*}
      \end{frame}
      
      \begin{frame}{Pseudo Code}
        \lstinputlisting[language=pseudo,style=global,caption={pseudocode for rrt algorithm},label=lst:rrt_algorithm]{rrt_algorithm.pseudo}
      \end{frame}
      
%        \begin{frame}<presentation:0|handout:1>{General Procedure}
%          \begin{itemize}
%            \item start with $q_{init}$ and $T$, with $K$ vertices
%            \item in each iteration a random state, $alpha$, is selected from $C$
%            \item find the closest vertex to $alpha$ with the terms of a distance metric
%            \item select an input that minimizes the distance from closest vertex to $alpha$ and check that the state is in $C_{free}$
%            \item is it free, add this new state as a vertex to $T$
%          \end{itemize}
%        \end{frame}
      
      \begin{frame}{Function: generate random state}
        \begin{itemize}
          \item generate a random state between the minimum and the maximum configuration limits of the robot
        \end{itemize}
        \lstinputlisting[language=pseudo,style=global,caption={pseudocode for random state generation},label=lst:generate_random_state]{generate_random_state.pseudo}
      \end{frame}



% tim's stuff

\begin{frame}{Function: find nearest neighbor}
	\begin{figure}[h]
		\footnotesize
		\begin{columns}
			\column[c]{.01\textwidth}
			\vspace{-50pt}
			a)
			\column[c]{0.49\textwidth}
			\includegraphics[width=1\textwidth]{4/rrt_nearest_neighbor_1.png}
			\column[c]{0.01\textwidth}
			\vspace{-50pt}
			b)
			\column[c]{0.49\textwidth}
			\includegraphics[width=1\textwidth]{4/rrt_nearest_neighbor_4.png}
		\end{columns}
		\vspace{15pt}
		\begin{columns}
			\column[c]{.25\textwidth}
			\column[c]{.01\textwidth}
			\vspace{-50pt}
			c)
			\column[c]{.49\textwidth}
			\includegraphics[width=1\textwidth]{4/rrt_nearest_neighbor_2.png}
			\column[c]{.25\textwidth}
		\end{columns}
		\caption{nearest neighbor [2]}
		\label{fig:generate state methode}
	\end{figure}
	\vspace{-35pt}
%	\begin{itemize}
%		\item bla
%	\end{itemize}
\end{frame}

\begin{frame}{Function: generate state}
	\begin{itemize}
		\item checked the path between $q_n$ and $alpha$ of collisions free
		\item if it collisions, then give back the last state before the collision
		\item if the time for trajectory larger as $delta\_time$, then give back the state at $delta\_time$
		\item else give $alpha$ back
	\end{itemize}
	
	\begin{figure}[h]
		\includegraphics[width=0.8\textwidth]{4/rrt_generate_state.png}
		\caption{generate state methode [2]}
		\label{fig:generate state methode}
	\end{figure}
\end{frame}

\begin{frame}{Function: generate state}
	\begin{figure}[h]
		\centering
		\includegraphics[width=0.8\textwidth]{4/rrt_generate_state.png}
		\caption{generate state methode [2]}
		\label{fig:generate state methode}
	\end{figure}
\end{frame}

\begin{frame}{graph buildup}
	\begin{figure}[h]
		\centering
		\includegraphics[width=0.5\textwidth]{4/rrt_history_top_down.jpg}
		\caption{todo}
	\end{figure}
\end{frame}

\begin{frame}
	\begin{figure}
		\centering
		\includemovie{0.8\textwidth}{0.8\textheight}{4/rrt_history_1200.mp4}
		\caption{graph aufbau}
	\end{figure}
\end{frame}

\begin{frame}
	\begin{figure}[h]
		\centering
		\includegraphics[width=0.5\textwidth]{4/rrt_history_top_down_graph.jpg}
		\caption{todo}
	\end{figure}
\end{frame}

\begin{frame}
	\begin{figure}[h]
		\centering
		\includegraphics[width=0.5\textwidth]{4/rrt_history_top_down_fusion.jpg}
		\caption{todo}
	\end{figure}
\end{frame}

\begin{frame}{Live Demo}
	\centering\Huge{Live Demo}
\end{frame}

\begin{frame}{huge example}
	\begin{figure}[h]
		\centering
		\includegraphics[width=0.6\textwidth]{4/rrt_history_1200.jpg}
		\caption{todo}
	\end{figure}
\end{frame}

\section{}
\begin{frame}{Sources}
	\begin{itemize}
		\item[1] Rapidly-Exploring Random Trees: A New Tool for Path Planning - Steven M. LaValle
		\\\url{http://coitweb.uncc.edu/~xiao/itcs6151-8151/RRT.pdf} (03.02.2015)
		\item[2] Planning Algorithms - Steven M. LaValle
		\\\url{http://planning.cs.uiuc.edu/} (03.02.2015)
		\item[3] Wikipedia - Rapidly-Exploring Random Tree
		\\\url{http://en.wikipedia.org/wiki/Rapidly_exploring_random_tree} (03.02.2015)
	\end{itemize}
\end{frame}

\begin{frame}
	\centering\Huge{Thank you for your attention!}
\end{frame}

\end{document}

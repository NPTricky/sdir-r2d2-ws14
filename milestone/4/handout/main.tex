\RequirePackage[l2tabu,orthodox]{nag} % first line of preamble (!)

\documentclass[%
  professionalfonts,%
  xcolor={%
    usenames,%
    dvipsnames,%
    svgnames,%
    table,%
    hyperref%
  }%
]{beamer}

\mode<presentation>
{
  \usetheme{Warsaw}
  \usecolortheme{seahorse}
  \useinnertheme{circles}
  \useoutertheme{smoothbars}
  \usefonttheme{professionalfonts} % clean display of mathematical letters
  \setbeamertemplate{navigation symbols}{} % no navigation symbols
  \addtobeamertemplate{navigation symbols}{}{%
  	\usebeamerfont{footline}%
  	\usebeamercolor[fg]{footline}%
  	{\large\insertframenumber/\inserttotalframenumber}%
  }
}

% encoding & decoding
\usepackage[T1]{fontenc}            
\usepackage[utf8]{inputenc}
\usepackage[english]{babel}

% math
\usepackage{isomath}
\usepackage{amsmath,amssymb,amsthm}
\usepackage{thmtools}
\usepackage{mathtools}

% fonts
%\usepackage{lmodern}
%\usepackage{mathpazo}
%\usepackage{kpfonts}
%\usepackage{mathptmx}
%\usepackage{stix}
%\usepackage{txfonts}
\usepackage{newtxtext,newtxmath}
%\usepackage{libertine} \usepackage[libertine]{newtxmath}

% figures & graphics
\usepackage{graphicx}
\usepackage[%
  usenames,%
  dvipsnames,%
  svgnames,%
  table,%
  hyperref%
]{xcolor}
\usepackage[%
  margin=10pt,%
  format=hang,%
  parskip=5pt,%
  singlelinecheck=false%
]{subfig}
\graphicspath{{content/}{../content/}{../../content/}{../../../content/}{../../../../content/}}
\DeclareGraphicsExtensions{.pdf,.png,.jpg,.gif}

% videos & medien
\usepackage{media9}%
\addmediapath{content/}
\addmediapath{../content/}
\addmediapath{../../content/}
\addmediapath{../../../content/}
\addmediapath{../../../../content/}

\newcommand{\includemovie}[3]{%
{%
\centering%
\includemedia[%
  width=#1,
  height=#2,%
  activate=pagevisible,%
  deactivate=pageclose,%
  addresource=#3,%
  flashvars={%
    src=#3 % same path as in addresource!
    &autoPlay=true % default: false; if =true, automatically starts playback after activation (see option ‘activation)’
    &loop=false % if loop=true, media is played in a loop
    &controlBarAutoHideTimeout=true %  time span before auto-hide
    &muted=true
  }%
]{}{StrobeMediaPlayback.swf}%
}%
}

% tables
\usepackage{booktabs}

% code listings
\usepackage{listings}
\lstdefinestyle{global}{ %
	basicstyle=\ttfamily\scriptsize\color{Black!90}, %
	stringstyle=\itshape\color{Magenta}, %
	showstringspaces=false, %
	keywordstyle={[1]\bfseries\color{MidnightBlue}}, %
	keywordstyle={[2]\bfseries\color{RedOrange}}, %
	keywordstyle={[3]\bfseries\color{Black!90}}, %
	commentstyle=\slshape\color{Blue}, %
	backgroundcolor=\color{Black!2}, %
	numbers=left, % where to put the line-numbers
  numberstyle=\footnotesize, % the size of the fonts that are used for the line-numbers
  stepnumber=1, % the step between two line-numbers. If it is 1 each line will be numbered
  numbersep=6pt, % how far the line-numbers are from the code
  frame=bt,%
  showspaces=false, % show spaces adding particular underscores
  showstringspaces=false, % underline spaces within strings
  showtabs=false, % show tabs within strings adding particular underscores
  rulesepcolor=\color{Black},
  rulecolor=\color{Black!60},
  tabsize=2, % sets default tabsize to 2 spaces
  captionpos=b, % sets the caption-position to bottom
  breaklines=true, % sets automatic line breaking
  breakatwhitespace=false, % sets if automatic breaks should only happen at whitespace
}

\lstdefinelanguage{pseudo}
{
  morekeywords=[1]{return, if, else, true, false, foreach, in}, %
  morekeywords=[2]{generate_rt, generate_random_state, find_nearest_neighbor,generate_state}, %
  morekeywords=[3]{}, %
  sensitive=false, %
  morecomment=[l]{//}, %
  morecomment=[s]{/*}{*/}, %
}

% microtype - last line of preamble
\usepackage{microtype}

% titlepage
\title{RRT - Rapidly-Exploring Random Trees}
\author{Tim Jagla, André Keuns, Anne Reich}
\institute[FIN]{Otto-von-Guericke-Universität Magdeburg}
\date{\today} 
%\logo{\pgfimage[width=2cm,height=2cm]{}}

\begin{document}
  \begin{frame}
    \titlepage
  \end{frame}
  
  \section{Collision Free Path Planning}
    \begin{frame}
      \huge{\centering{Collision Free Path Planning}}
    \end{frame}
    
    \subsection*{}
    \begin{frame}{Motivation}
      \begin{itemize}
        \item path planning: find a path from location \emph{A} to \emph{B}
        \item Example for path planning:
        \begin{itemize}
          \item mobile robot inside a build
          \item shall go to location XY
        \end{itemize}
        \item Example extension for collision free path planning:
        \begin{itemize}
          \item avoiding walls and not falling stairs
        \end{itemize}
      \end{itemize}
      
      % bild 0
      
    \end{frame}
    
    \subsection*{}
    \begin{frame}{Simple Example}
      \begin{itemize}
        \item Simple General Forward Search
        \begin{itemize}
          \item State: Unvisited, Dead, Alive
          \item Priority queue, \emph{Q}, with the set of alive states
          \item Start loop over \emph{Q}
          \item In each while iteration check next state
          \begin{itemize}
            \item It is the goal, is terminate
            \item Otherwise, it tries applying every possible action
           \end{itemize} 
        \end{itemize}
      \end{itemize}
      
      % bild 1
      
    \end{frame}
    
    \begin{frame}{Algorithms}
      \begin{itemize}
        \item Other known collsion free path planning algorithms
        \begin{itemize}
          \item Breadth first
          \item Deep first
          \item Dijikstra‘s algorithm
          \item A*
          \item Best First
          \item Backward search
          \item ...
        \end{itemize}
      \end{itemize}
    \end{frame}

    \begin{frame}{Principles}
      \begin{itemize}
        \item Basic Ingredients of Planning
        \begin{itemize}
          \item State
          \item Input
          \item Initial and goal states
          \item A criterion: Feasibility and/or Optimality
          \item a plan
        \end{itemize}
      \end{itemize}
    \end{frame}
    
    \section{Rapidly-Exploring Random Trees}
      \begin{frame}
        \huge{\centering{Rapidly-Exploring Random Trees}}
      \end{frame}
      
      \begin{frame}{Principles}
        \begin{itemize}
          \item Grows a tree tooted at the starting configuration by using random samples from the search space
          \item As each sample is drawn, a connection is attempted between it an the nearest state in the tree
          \item If the connection is feasible, this results in the addition of the new state to the tree
          \item The probability of expanding an existing state is proportional to the size of ist Voronoi region
          \begin{itemize}
            \item As the largest Voronoi regions belong to the states on the frontier of the search = the tree preferentially expands towards large unsearched areas
          \end{itemize}
        \end{itemize}
      \end{frame}
      
      \subsection*{}
        \begin{frame}{Notations}
          \begin{align*}
          X        & \text{ = metric space}\\
          x_{init} & \text{ = initale state}\\
          X_{goal} & \text{ = goal region},  X_{goal} \subset X\\
          C        & \text{ = configuration space of a rigid body or}\\
          & \text{ ~~ systems of bodies in a world}\\
          T        & \text{ = RRT (Tree of vertices)}\\
          T(C)     & \text{ = tanget bundle of the configuration space}\\
          X_{obs}  & \text{ = obstacle region},  X_{obs} \subset X\\
          X_{free} & \text{ = region without obstacles}, X_{free} \subset X\\
          \end{align*}
          Each edge of the RRT will correspond to a path that lies entirely in $X_{free}$
        \end{frame}
      
      \subsection*{}
      \begin{frame}{General Procedure}
        \begin{itemize}
          \item start with $x_init$ and $T$, with $K$ vertices
          \item in each iteration a random state, $x_rand$, is selected from $X$
          \item find the closest vertex to $x_rand$ with the terms of a distance metric
          \item select an input that minimizes the distance from closest vertex to $x_rand$ and check that the state is in $X_{free}$
          \item is it free, add this new state as a vertex to $T$
        \end{itemize}
      \end{frame}
      
      \subsection*{}
      \begin{frame}{Nice Properties}
        \begin{itemize}
          \item The expansion is heavily biased toward unexplored portions of the state space
          \item The distribution of vertices approaches the sampling distribution, leading to consistent behavior
          \item Is probabilistically complete under very general conditions
          \item sThe algorithm is relatively simple, which facilitates performance analysis
        \end{itemize}
      \end{frame}
      
      \begin{frame}{Nice Properties}
        \begin{itemize}
          \item It always remains connected, even though the number of edges is minimal
          \item Can be considered as a plat planning module, which can be adapted and incorporated into a wide variety of planning systems
          \item Entire plath planning algorithms can be constructed without requiring the ability to steer the system between two prescibed states, which greatly broadens the applicability of RRTs
        \end{itemize}
      \end{frame}
      
      \begin{frame}{Challanges of our work}
        überlegung 2d -> 3d -> 6d / configuration space
      \end{frame}
      
      \begin{frame}{Pseudo Code}
        
      \end{frame}
      
      \begin{frame}{Live Demo}
        graph aufbau 
        vollständigen graphen
      \end{frame}
        
  \section{}
    \begin{frame}{Sources}
      \begin{itemize}
        \item[1] Rapidly-Exploring Random Trees: A New Tool for Path Planning - Steven M. LaValle 
        \\\url{http://coitweb.uncc.edu/~xiao/itcs6151-8151/RRT.pdf} (03.02.2015)
        \item[2] Planning Algorithms - Steven M. LaValle \\\url{http://planning.cs.uiuc.edu/} (03.02.2015)
        \item[3] \url{http://en.wikipedia.org/wiki/Rapidly_exploring_random_tree} (03.02.2015)
    \end{itemize}
  \end{frame}
  
  \begin{frame}
    \Huge{\centering{Thank you for your attention}}
  \end{frame}

\end{document}

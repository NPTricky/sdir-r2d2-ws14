\RequirePackage[l2tabu,orthodox]{nag} % first line of preamble (!)

\documentclass[%
  professionalfonts,%
  xcolor={%
    usenames,%
    dvipsnames,%
    svgnames,%
    table,%
    hyperref%
  }%
]{beamer}

\mode<presentation>
{
  \usetheme{Warsaw}
  \usecolortheme{seahorse}
  \useinnertheme{circles}
  \useoutertheme{smoothbars}
  \usefonttheme{professionalfonts} % clean display of mathematical letters
  \setbeamertemplate{navigation symbols}{} % no navigation symbols
  \addtobeamertemplate{navigation symbols}{}{%
  	\usebeamerfont{footline}%
  	\usebeamercolor[fg]{footline}%
  	{\large\insertframenumber/\inserttotalframenumber}%
  }
}

% encoding & decoding
\usepackage[T1]{fontenc}            
\usepackage[utf8]{inputenc}
\usepackage[english,ngerman]{babel}

% math
\usepackage{isomath}
\usepackage{amsmath,amssymb,amsthm}
\usepackage{thmtools}
\usepackage{mathtools}

% fonts
%\usepackage{lmodern}
%\usepackage{mathpazo}
%\usepackage{kpfonts}
%\usepackage{mathptmx}
%\usepackage{stix}
%\usepackage{txfonts}
\usepackage{newtxtext,newtxmath}
%\usepackage{libertine} \usepackage[libertine]{newtxmath}

% figures & graphics
\usepackage{graphicx}
\usepackage[%
  usenames,%
  dvipsnames,%
  svgnames,%
  table,%
  hyperref%
]{xcolor}
\usepackage[%
  margin=10pt,%
  format=hang,%
  parskip=5pt,%
  singlelinecheck=false%
]{subfig}
\graphicspath{{content/}{../content/}{../../content/}{../../../content/}{../../../../content/}}
\DeclareGraphicsExtensions{.pdf,.png,.jpg,.gif}

% videos & medien
\usepackage{media9}%
\addmediapath{content/}
\addmediapath{../content/}
\addmediapath{../../content/}
\addmediapath{../../../content/}
\addmediapath{../../../../content/}

\newcommand{\includemovie}[3]{%
{%
\centering%
\includemedia[%
  width=#1,
  height=#2,%
  activate=pagevisible,%
  deactivate=pageclose,%
  addresource=#3,%
  flashvars={%
    src=#3 % same path as in addresource!
    &autoPlay=true % default: false; if =true, automatically starts playback after activation (see option ‘activation)’
    &loop=false % if loop=true, media is played in a loop
    &controlBarAutoHideTimeout=true %  time span before auto-hide
    &muted=true
  }%
]{}{StrobeMediaPlayback.swf}%
}%
}

% tables
\usepackage{booktabs}

% code listings
\usepackage{listings}
\lstdefinestyle{global}{ %
	basicstyle=\ttfamily\scriptsize\color{Black!90}, %
	stringstyle=\itshape\color{Magenta}, %
	showstringspaces=false, %
	keywordstyle={[1]\bfseries\color{MidnightBlue}}, %
	keywordstyle={[2]\bfseries\color{RedOrange}}, %
	keywordstyle={[3]\bfseries\color{Black!90}}, %
	commentstyle=\slshape\color{Blue}, %
	backgroundcolor=\color{Black!2}, %
	numbers=left, % where to put the line-numbers
  numberstyle=\footnotesize, % the size of the fonts that are used for the line-numbers
  stepnumber=1, % the step between two line-numbers. If it is 1 each line will be numbered
  numbersep=6pt, % how far the line-numbers are from the code
  frame=bt,%
  showspaces=false, % show spaces adding particular underscores
  showstringspaces=false, % underline spaces within strings
  showtabs=false, % show tabs within strings adding particular underscores
  rulesepcolor=\color{Black},
  rulecolor=\color{Black!60},
  tabsize=2, % sets default tabsize to 2 spaces
  captionpos=b, % sets the caption-position to bottom
  breaklines=true, % sets automatic line breaking
  breakatwhitespace=false, % sets if automatic breaks should only happen at whitespace
}

\lstdefinelanguage{pseudo}
{
  morekeywords=[1]{return, if, else, true, false, foreach, in}, %
  morekeywords=[2]{generate_rt, generate_random_state, find_nearest_neighbor,generate_state}, %
  morekeywords=[3]{}, %
  sensitive=false, %
  morecomment=[l]{//}, %
  morecomment=[s]{/*}{*/}, %
}

% microtype - last line of preamble
\usepackage{microtype}

% titlepage
\title[Singularity, Velocity Profiles and Trajectory Interpolation]{Singularity, Velocity Profiles and \\ Trajectory Interpolation}
\author{Tim Jagla, André Keuns, Anne Reich}
\institute[FIN]{Otto-von-Guericke-Universität Magdeburg}
\date{\today} 
%\logo{\pgfimage[width=2cm,height=2cm]{}}

\begin{document}
\begin{frame}
\titlepage
\end{frame}
  
\section{Singularities}
\begin{frame}{Singularities}
\begin{itemize}
\item unpredictable robot motion
\item infinite solutions
\item two types of singularity
  \begin{itemize}
  \item wrist singularity
  \item overhead singularity
  \end{itemize}
\end{itemize}
\end{frame}

\subsection{recap}
\begin{frame}
\begin{columns}
  \column[c]{.50\textwidth}
  \begin{figure}[h]
  \includegraphics[scale=0.65]{2/cad-robot.png}
  \caption{cad-robot}
  \end{figure}
  \column[c]{.50\textwidth}
  \begin{align*}
    \text{Joints:} &~ \text{Axis 1} = A1, \text{Axis 2} = A2, ...\\
    \theta1 = &~ \text{angle from $A1$}\\
    T_{0}^{4} = &~ \text{transformation matrix from $A4$}\\
    &~ \text{in world coordinate system}\\
    T_{0}^{4} = &~ T_{0}^{1} \cdot T_{1}^{2} \cdot T_{2}^{3} \cdot T_{3}^{4}
  \end{align*}
\end{columns}
\end{frame}

\subsection{Wrist singularity}
\begin{frame}
\includemovie{0.8\textwidth}{0.8\textheight}{2/singularities_video_wrist.mp4}
\end{frame}

\begin{frame}
\begin{itemize}
\item Appearance:
  \begin{itemize}
  \item $A4$ and $A6$ are collinear together
  \end{itemize}
  \item Mathematics:
    \begin{itemize}
    \item $T_{0}^{4} = T_{0}^{1} \cdot T_{1}^{2} \cdot T_{2}^{3} \cdot T_{3}^{4}$         
    \item $T_{0}^{6} = T_{0}^{TCP} \cdot (T_{0}^{Tool})^{-1}$
    \item Check collinear:
    \begin{itemize}
    \item take z-vector of each matrix
    \item if $\vec{v_1} \cdot \mu = \vec{v_2}$ then is \emph{wrist singularity}
    \end{itemize}
  \end{itemize}
\end{itemize}
\end{frame}

\begin{frame}[fragile]
\lstinputlisting[language=pseudo,style=global,caption={pseudocode for detection of a wrist singularity},label=lst:wristsingularity]{isWristSingularity.pseudo}
\end{frame}

\subsection{Overhead singularity}
\begin{frame}
\includemovie{0.8\textwidth}{0.8\textheight}{2/singularities_video_overhead.mp4}
\end{frame}

\begin{frame}
\begin{itemize}
\item Appearance:
  \begin{itemize}
  \item wrist-point above $A1$
  \end{itemize}
\item Mathematics:
  \begin{itemize}
  \item $T_{0}^{WP} = T_{0}^{TCP} \cdot (T_{0}^{Tool})^{-1}$
  \item           
  $\begin{pmatrix}
  a_1 & a_2 & a_3 & p_1 \\
  b_1 & b_2 & b_3 & p_2 \\
  c_1 & c_2 & c_3 & p_3 \\
  0   & 0   & 0   & 1 
  \end{pmatrix}$, 
  if $p_1 = p_2 = 0$ then \emph{overhead singularity}
  \end{itemize}
\end{itemize}  
\end{frame}

\begin{frame}[fragile]
\lstinputlisting[language=pseudo,style=global,caption={pseudocode for detection of an overhead singularity},label=lst:overheadsingularity]{isOverheadSingularity.pseudo}
\end{frame}

\subsection{solution of the entire singularity problem}
\begin{frame}[fragile]
\lstinputlisting[language=pseudo,style=global,caption={pseudocode for check  the singularity},label=lst:singularitysolution]{singularitySolution.pseudo}
\end{frame}

%\part{Motion Planning}
%The motion planning is divided in the velocity profiles and the trajectory interpolation

\section{Velocity Profiles}
\begin{frame}{Velocity Profiles}
  \begin{columns}
	\column[c]{.2\textwidth}
	\renewcommand{\arraystretch}{3.5}
	\begin{tabular}{c}
		Async. \\ 
		Sync. \\ 
		Full Sync.
	\end{tabular} 
    \column[c]{.4\textwidth}
    \begin{figure}[h]
      \includegraphics[width=\columnwidth]{2/vp-ptp-motion.png}
      \label{fig:vp-ptp-motion}
    \end{figure}
    \column[c]{.4\textwidth}
    \begin{figure}[h]
      \includegraphics[width=\columnwidth]{2/vp-ptp-motion_challenge.png}
      \label{fig:vp-ptp-motion_challenge}
    \end{figure}
  \end{columns}
\end{frame}

\subsection{Asynchronous PTP}
%handout
\begin{frame}<presentation:0|handout:1>
\begin{itemize}
\item for each joint i their current state, target state and their acceleration maximum $\hat{a}_{m,i}$ are given
\item the total motion time $t_{e,i}$ are the output of velocity profile calculation
\item the waypoints $X_{i}[n_{i}]$ are the output motion planning
\end{itemize}
\end{frame}
\subsubsection{Step 1}
\begin{frame}<presentation:0|handout:1>
determine maximum amplitude for velocity ramp
\begin{itemize}
\item check whether the distance $\hat{x}_{e,i}$ is sufficient to let the velocity grow up to $\hat{v}_{m,i}$ or not
\item if we run into the velocity triangle situation we must limit $\hat{v}_{m,i}$ down to $v_{m,i}$
\end{itemize}
\end{frame}

\begin{frame}
\begin{align*}
\hat{x}_{e,i} & = \sqrt{{\Delta\hat{x}_{e,x}}^{2} + {\Delta\hat{x}_{e,y}}^{2} + {\Delta\hat{x}_{e,z}}^{2} } \\
v_{m,i } & = min(\hat{v}_{m,i},\sqrt{\hat{x}_{e,i} \cdot \hat{a}_{m,i}})  \label{eq:asynvmi}
\end{align*}

\begin{align*}
v_{m} & \text{ is the velocity ramp maximum} \\
\hat{v}_{m} & \text{ is the predetermined velocity ramp maximum} \\
\hat{x}_{e} & \text{ is the predetermined distance to move} \\
\hat{a}_{m} & \text{ is the predetermined acceleration maximum} 
\end{align*}
\end{frame}

\subsubsection{Step 2}
\begin{frame}
compute the acceleration duration $t_{a,i}$, the deceleration start time $t_{d,i}$ and the total motion time $t_{e,i}$
\begin{align*}
t_{a,i} & = \frac{v_{m,i}}{\hat{a}_{m,i}} \\
t_{e,i} & = \frac{\hat{x}_{e,i}}{v_{m,i}} + t_{a,i} \\
t_{d,i} & = t_{e,i} - t_{a,i}
\end{align*}

\begin{align*}
t_{a} & \text{ is the acceleration length} \\
t_{e} & \text{ is the total motion time} \\
t_{d} & \text{ is the deceleration start time}
\end{align*}
\end{frame}
  
\subsection{Synchronous PTP}
\begin{frame}
the calculation is the same as the asynchronous PTP
exept that all motion durations coincide to $t_{e,max}$

\begin{equation*}
t_{e,1} = t_{e,1} = ... = t_{e,N} = t_{e,max}
\end{equation*}
\end{frame}

\subsubsection{Step 1}
\begin{frame}<presentation:0|handout:1>
the first step is the same as step 1 and 2 in the asynchronous PTP calculation. we determine the maximum velocity ramp and  calculate the time of acceleration
\end{frame}

\begin{frame}
\begin{align*}
\tilde{v}_{m,i} & = min(\hat{v}_{m,i},\sqrt{\hat{x}_{e,i} \cdot \hat{a}_{m,i}}) \\
\tilde{t}_{a,i} & = \frac{v_{m,i}}{\hat{a}_{m,i}}\\
t_{e,i} & = \frac{\hat{x}_{e,i}}{v_{m,i}} + t_{a,i}
\end{align*}

\begin{align*}
\tilde{v}_{m} & \text{ is the interim velocity ramp maximum}\\
\tilde{t}_{a} & \text{ is the interim acceleration length}\\
\end{align*}                 
\end{frame}

\begin{frame}
the total movement time $t_{e,max}$ is the maximum value of all movements
\begin{equation*}
t_{e,max} = max(t_{e,1},t_{e,2},...,t_{e,N})
\end{equation*}

\begin{align*}
t_{e,N} & \text{ is the total motion time } \\
t_{e,max} & \text{ is the maximum value of the total motion time}      
\end{align*}
\end{frame}
  
\subsubsection{Step 2}
\begin{frame}
calculation of the maximum velocities of each axis $v_{m,i}$

\begin{equation*}
t_{e,max} = \frac{\hat{x}_{e,i}}{\hat{v}_{m,i}}+\frac{\hat{v}_{m,i}}{\hat{a}_{m,i}}
\end{equation*}

%over conversion of the formula we get the formula for velocity of each axis
%over the p-q-formula

\begin{equation*}
v_{m,i} = \frac{t_{e,max} \cdot \hat{a}_{m,i}}{2}-\sqrt{\frac{t_{e,max}^2 \cdot \hat{a}_{m,i}^2}{4}-\hat{a}_{m,i}\cdot \hat{x}_{e,i}}
\end{equation*}
\end{frame}
  
\subsubsection{Step 3}
\begin{frame}
calculation of the acceleration duration $t_{a,i}$

\begin{equation*}
t_{a,i} = \frac{\hat{v}_{m,i}}{\hat{a}_{m,i}}
\end{equation*}

\begin{equation}
t_{d,max} = t_{e,max}-t_{a,i}
\end{equation}      

\begin{align*}
t_{a} & \text{ is the acceleration length} \\
t_{e} & \text{ is the total motion time} \\
t_{d} & \text{ is the deceleration start time}
\end{align*}
\end{frame}

\section{Trajectory Interpolation}
\subsection{Types of Interpolation}
\begin{frame}{Trajectory Interpolation}
the interpolation can divide in three classes.
\begin{itemize}
\item the \emph{linear interpolation} is very simple, they describes the direct movement from point \emph{A} to point \emph{B}.
\item the \emph{circular interpolation} is a little bit tricky, they describes the a circular movement form point \emph{A} to point \emph{B} over a point \emph{C}.
\item the \emph{spline interpolation} is very complex, they divides the trajectory into individual parts, each part is describes by a higher-order polynomial.
\end{itemize}
\end{frame}

\subsection{Linear Interpolation PTP}
%\subsubsection{Step 1}
%\begin{frame}
%determine the count of waypoints
%
%\begin{align}
%n_{a,i} & = & \frac{t_{a,i}}{h} \\
%n_{e,i} & = & \frac{t_{e,i}}{h}
%\end{align}

%\begin{align*} 
%n_{a} & \text{ is the number of time slices for acceleration phase} \\
%n_{e} & \text{ is the number of time slices for moment phase} \\
%h & \text{ is the period time (samplingrate)}
%\end{align*}
%\end{frame}

\subsubsection{Step 1}
\begin{frame}
\begin{itemize}
  \item calculation of the distance covered of each axes
\end{itemize}

\begin{figure}[h]
  \includegraphics[width=\columnwidth]{2/one-axis-acceleration-profile.png}
  \caption{one-axis-acceleration-profile}
  \label{fig:ti-movement-cases}
\end{figure}

\begin{itemize}
\item case 1: \emph{acceleration phase} $t \leq t_{a}$
\item case 2: \emph{constant movement phase} $t_{a} \le t \leq t_{e} - t_{a}$
\item case 3: \emph{deceleration phase} $t_{e} - t_{a} \le t \leq t_{e}$
\item case 4: \emph{stood phase} $t_{e} \le t$
\end{itemize}

\end{frame}

\begin{frame}
\begin{equation*}
X_{i}(t) = 
\begin{dcases}
\frac{1}{2} \hat{a}_{m,i} \cdot t^2  & t \leq t_{a,i} \\
v_{m,i} \cdot t - \frac{1}{2} \cdot \frac{v_{m,i}^2}{\hat{a}_{m,i}} & t_{a,i} \le t \leq t_{e,i} - t_{a,i} \\
v_{m,i} \cdot (t_{e,i} - t_{a,i}) - \frac{\hat{a}_{m,i}}{2} \cdot (t_{e,i} - t)^2 & t_{e,i} - t_{a,i} \le t \leq t_{e,i}\\
\hat{x}_{e,i} & t_{e,i} < t\\
\end{dcases}
\end{equation*}

\begin{align*}
X_{i}(t) & \text{ distance covered at time $t$ } \\
\end{align*}
\end{frame}

\subsubsection{Step 2}
\begin{frame}
determine position / orientation to time $t$

\begin{equation*}
\vec{x}_{i}(t) = \vec{x}_{i}(0) + \frac{X_{i}(t)}{\hat{x}_{e,i}} \cdot \Bigl(\vec{x}_{i}(t_{e}) - \vec{x}_{i}(0) \Bigr)
\end{equation*}

\begin{align*}
\vec{x}_{i}(0)     & \text{ is the beginning position/orientation of a point} \\
\vec{x}_{i}(t_{e}) & \text{ is the end position/orientation of a point} \\
\end{align*}
\end{frame}

\subsection{Linear Interpolation CP}
\begin{frame}
\begin{columns}
  \column[c]{.50\textwidth}
  \begin{equation*}
  \vec{x}_A = \left( \begin{array}{c} P_{x,A}\\ P_{y,A}\\ P_{z,A}\\ \hline \alpha_{x,A}\\ \beta_{y,A}\\ \gamma_{z,A} \end{array}\right) = \left( \begin{array}{c} P_{A}\\\hline\Theta_{A} \end{array}\right)
  \end{equation*}
  \column[c]{.50\textwidth}
  \begin{equation*}
  \vec{x}_B = \left( \begin{array}{c} P_{x,B}\\ P_{y,B}\\ P_{z,B}\\ \hline \alpha_{x,B}\\ \beta_{y,B}\\ \gamma_{z,B} \end{array}\right) = \left( \begin{array}{c} P_{B}\\\hline\Theta_{B} \end{array}\right)
  \end{equation*}
\end{columns}
\begin{align*}
P & \text{ is the position of a point in the world coordinate system} \\
\Theta & \text{ is the orientation of a point in the word coordinate system}   
\end{align*}
\end{frame}

\subsubsection{Step 1}
\begin{frame}
calculation of the path length $\hat{x}_{e,P}$ and the angle difference $\hat{x}_{e,\theta}$
\begin{align*}
\hat{x}_{e,P} & = \left|P_{B} - P_{A}\right|\\
& = \sqrt{(P_{x,B} - P_{x,A})^2 + (P_{y,B} - P_{y,A})^2 + (P_{z,B} - P_{z,A})^2 }\\
\\
\hat{x}_{e,\theta} & = \left|\Theta_{B} - \Theta_{A}\right|\\
& = \sqrt{(\Theta_{x,B} - \Theta_{x,A})^2 + (\Theta_{y,B} - \Theta_{y,A})^2 + (\Theta_{z,B} - \Theta_{z,A})^2 }
\end{align*}
\end{frame}

\subsubsection{Step 2}
\begin{frame}
calculation the duration of the motion for the path length $t_{e,P}$ and the angle difference $t_{e,\theta}$
\begin{align*}
t_{e,P} & = \frac{\hat{x}_{e,P}}{\hat{v}_{P}}+\frac{\hat{v}_{P}}{\hat{a}_{P}} \\
\\
t_{e,\theta} & = \frac{\hat{x}_{e,\theta}}{\hat{v}_{\theta}}+\frac{\hat{v}_{\theta}}{\hat{a}_{\theta}}
\end{align*}
the speed $\hat{v}$ and the acceleration $\hat{a}$ is set by the operator
\end{frame}

\subsubsection{Step 3}
\begin{frame}
Synchronisation of the motion
\begin{equation*}
t_{e, max} = max\left(t_{e,P}, t_{e,\theta}\right)
\end{equation*}
\begin{itemize}
\item $t_{e,\theta} \le t_{e,P}$:
\begin{align*}
v & = \frac{t_{e,max} \cdot \hat{a}_{e}}{2}-\sqrt{\frac{t_{e,max}^2 \cdot \hat{a}_{e}^2}{4}-\hat{a}_{e}\cdot \hat{x}_{e}} \\
t_{a} & = \frac{v}{\hat{a}_{e}}
\end{align*}
\item $t_{e,P} \leq t_{e,\theta}$:
\begin{align*}
v & = \frac{t_{e,max} \cdot \hat{a}_{\theta}}{2}-\sqrt{\frac{t_{e,max}^2 \cdot \hat{a}_{\theta}^2}{4}-\hat{a}_{\theta}\cdot \hat{x}_{\theta}} \\
t_{a} & = \frac{v }{\hat{a}_{\theta}}
\end{align*}
\end{itemize}
\end{frame}

\subsubsection{Step 4}
\begin{frame}
the position $P(t)$ to time $t$
\begin{align*}
P(t) & = P_{A} + \frac{X_{P}(t)}{\hat{x}_{e,P}} \cdot \left(P_{B} - P_{A} \right)\\
\end{align*}
the orientation $\Theta(t)$ to time $t$
\begin{align*}
\Theta(t) & = \Theta_{A} + \frac{X_{\Theta}(t)}{\hat{x}_{e,\Theta}} \cdot \left(\Theta_{B} - \Theta_{A} \right)\\
\end{align*}
\end{frame}


\subsection{Determine the Configuration form the inverse Kinematic}
\begin{frame}
one way is, we determine the configuration on hand of the minimum angle form an axis. e.g. from angle 4 \\


an another way is the determination of the configuration under consideration of each angle from a configuration.\\
for a better result, we given every angle a weighting for the calculation.\\
for energy efficiency movement we determine the weighting for each angle on hand the power of the engine of any angle\\

\end{frame}

\section{Example for Synchronous PTP} 

\subsection{Given Values }
\begin{frame}
we have a 3 arm robot with 3 links and 3 joints and following properties\\

\begin{align*}
h & = 10 ~ms \\
\hat{v}_{m} & = 200 ~^\circ/s \\
\hat{a}_{m} & = 100 ~^\circ/s^2
\end{align*}

\begin{center}
\begin{tabular}{ccccc}
\toprule
i & 0 & 1 & 2 & 3 \\
\midrule
$\alpha_{t_{0},i}$ & $  0^\circ$ & $20^\circ$ & $30^\circ$ & $-20^\circ$ \\ 
$\alpha_{t_{e},i}$ & $ 64^\circ$ & $11^\circ$ & $46^\circ$ & $  5^\circ$ \\
\bottomrule
\end{tabular}
\end{center}
\end{frame}

\subsection{Determine the interim velocity and the total movement time}
\begin{frame}
in the first step, we calculation for each axis the distance of movement, the interim velocity, the movement time of each axis and the total movement time

\subsubsection*{the distance of movement }
\begin{equation*}
\hat{X}_{e,i} = {\alpha_{t_{e}|i} - \alpha_{t_{0}|i}}
\end{equation*}

the result are following    
\begin{center}
\begin{tabular}{ccccc}
\toprule
i & 0 & 1 & 2 & 3 \\
\midrule
$\hat{X}_{e,i}$ & $ 64^\circ$ & $ -9^\circ$ & $16^\circ$ & $25^\circ$ \\
\bottomrule 
\end{tabular}
\end{center}
\end{frame}

\subsubsection*{the interim velocity}
\begin{frame}
the interim velocity $\tilde{v}_{m,i}$ of each axis is 
\begin{equation*}
\tilde{v}{m,i} = min\left(\hat{v}_{m,i},\sqrt{|\hat{x}_{e,i}| \cdot \hat{a}_{m,i}} \right) 
\end{equation*}

the values of the calculation are for each axis are
\begin{center}
\begin{tabular}{ccccc}
\toprule
i & 0 & 1 & 2 & 3 \\
\midrule
$\tilde{v}_{m,i}$ & $80^\circ/s$ & $30^\circ/s$ & $40^\circ/s$ & $50^\circ/s$ \\
\bottomrule
\end{tabular}               
\end{center}  
\end{frame}

\subsubsection*{the movement time of each axis}
\begin{frame}
before we determine the movement time of each axis, we need the interim acceleration time of each axis

\begin{align*}
\tilde{t}_{a,i} & = \frac{v_{m,i}}{\hat{a}_{m,i}} \\
t_{e,i} & = \frac{\hat{x}_{e,i}}{v_{m,i}} + t_{a,i} \\
\end{align*}

\begin{center}
\begin{tabular}{ccccc}
\toprule
i & 0 & 1 & 2 & 3 \\
\midrule
$\tilde{t}_{a|i}$ & 0.8s & 0.3s & 0.4s & 0.5s \\ 
$t_{e|i}$ & 1.6s & 0.6s & 0.8s & 1.0s \\
\bottomrule 
\end{tabular}
\end{center} 
\end{frame}

\subsubsection*{determine the total movement time}
\begin{frame}
\begin{equation*}
t_{e,max} = max(t_{e,1},t_{e,1},...,t_{e,N})
\end{equation*}

the biggest value of the movement time of all axis is the total movement time.
\begin{equation*}
t_{e,max} = 1.6s
\end{equation*}    

\subsection{Calculation of the velocity }
after the interim velocity and the total movement time, we can calculate the velocity value for each axis and the times for acceleration and deceleration

\subsubsection*{the velocity}
\begin{equation*}
v_{m,i} = \frac{t_{e,max} \cdot \hat{a}_{m,i}}{2}-\sqrt{\frac{t_{e,max}^2 \cdot \hat{a}_{m,i}^2}{4}-\hat{a}_{m,i}\cdot \hat{x}_{e,i}}
\end{equation*}

\begin{center}
\begin{tabular}{ccccc}
\toprule
i & 0 & 1 & 2 & 3 \\
\midrule
$v_{m|i}$  & $80.0^\circ/s$   & $5.8^\circ/s$   & $10.7^\circ/s$  & $17.5^\circ/s$ \\
\bottomrule 
\end{tabular}
\end{center}
\end{frame}

\subsection{Determine the acceleration and the deceleration time}
\begin{frame}
\begin{equation*}
t_{a,i} = \frac{\hat{v}_{m,i}}{\hat{a}_{m,i}}
\end{equation*}
\begin{equation*}
t_{d,j} = t_{e,max}-t_{a,i}
\end{equation*}    

\begin{center}
\begin{tabular}{ccccc}
\toprule
i & 0 & 1 & 2 & 3 \\
\midrule
$t_{a|i}$ & 0.800s & 0.058s & 0.170s & 0.175s \\ 
$t_{d|i}$ & 0.800s & 1.542s & 1.430s & 1.425s \\
\bottomrule
\end{tabular}
\end{center}
\end{frame}

\subsection{Determine the count of waypoints}
\begin{frame}
\begin{align}
n_{a,i} & = \frac{t_{a,i}}{h} \\
n_{e,max} & = \frac{t_{e,max}}{h}
\end{align}

the count of time slices for the movement phase is 
\begin{equation*}
n_{e,max} = 160
\end{equation*}

the count of time slices for the acceleration phase of each axis are
\begin{center}
\begin{tabular}{ccccc}
\toprule
i & 0 & 1 & 2 & 3 \\
\midrule
$n_{a|i}$ & 80 & 5.8 & 17 & 17.5 \\
\bottomrule
\end{tabular}
\end{center}
\end{frame}

\subsection{Calculation of the Waypoints}
\begin{frame}
\begin{equation*}
X_{i}[n_{i}] = 
\begin{dcases}
\frac{1}{2} \hat{a}_{m,i} \cdot h^2 \cdot n_{i}^2 & 1 \leq n_{i} \leq n_{a,i} \\ 
v_{m,i} \cdot h \cdot n_{i} - \frac{1}{2} \cdot \frac{v_{m,i}^2}{\hat{a}_{m,i}} & n_{a,i} < n_{i} \leq n_{e,i} - n_{a,i} \\
v_{m,i} \cdot (t_{e,i} - t_{a,i}) - \frac{\hat{a}_{m,i}}{2} \cdot (t_{e,i} - h \cdot n_{i})^2   & n_{e,i} - n_{a,i} < n_{i} \leq n_{e,i}
\end{dcases}
\end{equation*}
we calculation the waypoints (angles) at a time $n_i = 4$, $n_i = 80$ and $n_i = 156s$.
\end{frame}

\begin{frame}
in the acceleration phase of all axes $n_i = 4$:

\begin{center}
\begin{tabular}{ccccc}
\toprule
i & 0 & 1 & 2 & 3 \\
\midrule
$n_{a,i}$ & 80 & 5.8 & 17 & 17.5 \\ 
$1 \leq n_{i} \leq n_{a,i}$ & true & true & true & true \\
\bottomrule
\end{tabular}
\end{center}

we can see that we are for each axis in the acceleration phase. for the acceleration phase, we take the following formula

\begin{equation*}
X_{i}[n_{i}] = \frac{1}{2} \hat{a}_{m,i} \cdot h^2 \cdot n_{i}^2
\end{equation*}

\begin{center}
\begin{tabular}{ccccc}
\toprule
i & 0 & 1 & 2 & 3 \\
\midrule
$X_{i}[n_{i}]$ & 80 & 5.8 & 17 & 17.5 \\
\bottomrule
\end{tabular}
\end{center}
\end{frame}

\begin{frame}
in the movement phase without acceleration or deceleration $n_i = 80$

\begin{center}
\begin{tabular}{ccccc}
\toprule
i & 0 & 1 & 2 & 3 \\
\midrule
$n_{a,i}$ & 80 & 5.8 & 17 & 17.5 \\ 
$ n_{e,i} - n_{a,i}$ & 80 & 154.2 & 143 & 142.5 \\
$ n_{a,i} < n_{i} \leq n_{e,i} - n_{a,i}$ & false & true & true & true \\
\bottomrule
\end{tabular}
\end{center}

we can see that we are for each axis in the movement phase without acceleration or deceleration except for
axis 0, for this axis is it the end of the acceleration phase.

for this phase, we take the following formula

\begin{equation*}
X_{i}[n_{i}] = v_{m,i} \cdot h \cdot n_{i} - \frac{1}{2} \cdot \frac{v_{m,i}^2}{\hat{a}_{m,i}} 
\end{equation*}

\begin{center}
\begin{tabular}{ccccc}
\toprule
i & 1 & 2 & 3 \\
\midrule
$X_{i}[n_{i}]$ & $4.5^\circ$ & $8.0^\circ$ & $12.5^\circ$ \\ 
$\alpha_{t_{n}}$ & $15.5^\circ$ & $38.0^\circ$ & $-7.5^\circ$ \\
\bottomrule
\end{tabular}
\end{center}
\end{frame}

\begin{frame}
in the deceleration phase $n_i = 156$

\begin{center}
\begin{tabular}{ccccc}
\toprule
i & 0 & 1 & 2 & 3 \\
\midrule
$n_{a,i}$ & 80 & 5.8 & 17 & 17.5 \\ 
$n_{e,i} - n_{a,i}$ & 80 & 154.2 & 143 & 142.5 \\
$n_{e,i} - n_{a,i} < n_{i} \leq n_{e,i}$ & true  & true & true & true \\
\bottomrule 
\end{tabular}
\end{center}

we can see that we are for each axis in the deceleration phase

for the deceleration phase, we take the following formula

\begin{equation*}
X_{i}[n_{i}] = v_{m,i} \cdot (t_{e,i} - t_{a,i}) - \frac{\hat{a}_{m,i}}{2} \cdot (t_{e,i} - h \cdot n_{i})^2 
\end{equation*}

\begin{center}
\begin{tabular}{ccccc}
\toprule
i & 0 & 1 & 2 & 3 \\
\midrule
$X_{i}[n_{i}]$ & $63.9^\circ$ & $8.8^\circ$ & $15.2^\circ$ & $24.8^\circ$ \\
$\alpha_{t_{n}}$ & $63.9^\circ$ & $11.2^\circ$ & $45.2^\circ$ & $4.8^\circ$ \\
\bottomrule
\end{tabular}
\end{center} 
\end{frame}

\end{document}
